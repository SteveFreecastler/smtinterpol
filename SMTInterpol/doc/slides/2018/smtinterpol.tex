%%%%%%%%%%%%%%%%%%%%%%%%%%%%%%%%%%%%%%%%%%%%%%%%%%%%%%%%%%%%%%%%%%%%%%%%
%
%   Folien: SMT Workshop
%           SMTInterpol
%   Juli 2018
%
%%%%%%%%%%%%%%%%%%%%%%%%%%%%%%%%%%%%%%%%%%%%%%%%%%%%%%%%%%%%%%%%%%%%%%%%

\documentclass[table,aspectratio=169]{beamer}
\usepackage{xcolor}
\usepackage{tikz}
\usetikzlibrary{fit,shapes.callouts,shapes.geometric,shapes.symbols}
\usetikzlibrary{positioning,calc}
\usetheme{Madrid}
\usecolortheme{freiburg}
\let\checkmark\relax
\usepackage{dingbat}

\newcommand\store[3]{\ensuremath{#1\langle #2\lhd#3\rangle}}
\newcommand\select[2]{\ensuremath{#1[#2]}}
\newcommand\diff[2]{\ensuremath{\mathop{\mathrm{diff}}(#1,#2)}}

\geometry{papersize={16.80cm,12.60cm}}
\setbeamertemplate{navigation symbols}{}

%%% Titel, Autor und Datum des Vortrags:
\pgfdeclareimage[height=1cm]{unifr}{unifr-neu}

%% Institut
\institute[Uni Freiburg]{University of Freiburg\hspace{1cm}\pgfuseimage{unifr}}

\colorlet{colorTest}{ALUblue}

\useoutertheme{unifr}


\begin{document}

\section{Motivation}

\begin{frame}[plain,fragile]
  \pgfdeclarelayer{back}
  \pgfsetlayers{back,main}
  
\begin{center}
\begin{tikzpicture}
  \begin{scope}[xshift=-5cm,yshift=4.5cm]
    \node (qfla) at (.2,.4) {Quantifier Free};
    \node (la) at (0,0) {Linear Arithmetic};
    \node (la1) at (.5,-.6) {$y \leq i+1$};
    \node (la2) at (-.8,-1.1) {$i \leq y$ };
    \node (lamixed) at (0,-1.6) {$y - to\_int(y) < .3$};
  \end{scope}
  \begin{scope}[xshift=5cm,yshift=4.5cm]
    \node (qfuf) at (0,.4) {Quantifier Free};
    \node (uf) at (0,0) {Uninterpreted Functions};
    \node (uf1) at (-.6,-.6) {$f(b) = v$};
    \node (uf2) at (.7,-1.1) {$f(a) \neq v$ };
  \end{scope}
  \begin{scope}[xshift=0cm,yshift=5cm]
    \node (qfar) at (0,.6) {Quantifier Free};
    \node (ar) at (0,.2) {Arrays};
    \node (ar1) at (-.3,-.4) {$b = \store aiv$};
    \node (ar2) at (.3,-0.9) {$\select av = v$};
  \end{scope}
    
  \begin{scope}[xshift=0cm,yshift=4cm]
    \node (comb) at (0, -.9) {Theory combination};
    \node (lauf1) at (0, -1.5) {$\select b i \geq i$};
    \node (lauf2) at (-.7, -2) {$f(i+y) = 2v$};
  \end{scope}
  \begin{scope}[xshift=0cm,yshift=4cm]
    \node (lauf3) at (.3, -2.5) {$f(b) \leq i$};
  \end{scope}
        
  \node (url) at (0,-3)
    {\url{http://ultimate.informatik.uni-freiburg.de/smtinterpol}};
    
  \coordinate (a1) at (-6,3.6);
  \coordinate (a2) at (6.5,3.6);
  \node[shape=cloud callout,cloud puffs=40,draw,inner sep=-1cm,aspect=3,
    callout relative pointer={(0,-.4)},
    fit=(a1)(a2)
  ] {};
    
  \node (si) at (0,-.5) {\huge SMTInterpol};
    
  \node[draw,shape=rectangle callout, callout relative pointer={(1.7,-.3)}] at (-4,.1) {sat};
    
  \node[draw,shape=rectangle callout, callout relative pointer={(1.4,.2)}] (model) at (-4, -.9) {model};
    
  \node[draw,shape=rectangle callout, callout relative pointer={(-3.2,-.5)}] at (5.4,.5) {unsat};
    
  \node[shape=ellipse,inner sep=.01cm,thick](proof) at (5.4,-.35) {proof};
  \node[draw,shape=rectangle callout, fit=(proof), callout relative pointer={(-2.5,-.15)}] {};
    
  \node[shape=ellipse,inner sep=.03cm,thick](uc) at (5.4,-1.25) {unsat core};
  \node[draw,shape=rectangle callout, fit=(uc), callout relative pointer={(-1.8,.1)}] (ucco) {};
    
  \node (ips) at (0,-2) {interpolants};
  \node[draw,shape=rectangle callout, callout relative pointer={(0,+.8)}, fit=(ips)] {};
\end{tikzpicture}
\end{center}
\end{frame}
\end{document}
