%%%%%%%%%%%%%%%%%%%%%%%%%%%%%%%%%%%%%%%%%%%%%%%%%%%%%%%%%%%%%%%%%%%%%%%%
%
%   Folien: SMT Workshop
%           SMTInterpol
%   Juli 2020
%
%%%%%%%%%%%%%%%%%%%%%%%%%%%%%%%%%%%%%%%%%%%%%%%%%%%%%%%%%%%%%%%%%%%%%%%%

\documentclass[table,aspectratio=169]{beamer}
\usepackage{xcolor}
\usepackage{tikz}
\usetikzlibrary{fit,shapes.callouts,shapes.geometric,shapes.symbols}
\usetikzlibrary{positioning,calc}
\usetheme{Madrid}
\usecolortheme{freiburg}
\let\checkmark\relax
\usepackage{dingbat}

\newcommand\store[3]{\ensuremath{#1\langle #2\lhd#3\rangle}}
\newcommand\select[2]{\ensuremath{#1[#2]}}
\newcommand\diff[2]{\ensuremath{\mathop{\mathrm{diff}}(#1,#2)}}

\geometry{papersize={16.80cm,12.60cm}}
\setbeamertemplate{navigation symbols}{}

%%% Titel, Autor und Datum des Vortrags:
\pgfdeclareimage[height=1cm]{unifr}{unifr-neu}

%% Institut
\institute[Uni Freiburg]{University of Freiburg\hspace{1cm}\pgfuseimage{unifr}}

\colorlet{colorTest}{ALUblue}

\useoutertheme{unifr}


\begin{document}

\section{Motivation}

\begin{frame}[plain,fragile]
  \frametitle{SMTInterpol}
  \framesubtitle{J\"{u}rgen Christ, Jochen Hoenicke, Tanja Schindler}
  \pgfdeclarelayer{back}
  \pgfsetlayers{back,main}

  \begin{itemize}
  \item CDCL(T) based SMT solver
    \begin{tikzpicture}[baseline=-0.5cm]
      \node[draw,shape=rectangle callout, callout relative pointer={(-.7,-.1)}] at (0,0) {sat};
      \node[draw,shape=rectangle callout, callout relative pointer={(-1.1,-.2)}] at (1.6,-.1) {unsat};
    \end{tikzpicture}
  \item for Arrays, Uninterpreted Functions, Linear Integer and Real Arithmetic
    (AUFLIRA)\\
    \begin{tikzpicture}
  \begin{scope}[xshift=2cm,yshift=5cm]
    \node (la1) at (-.8,-.6) {$y \leq i+1$};
    \node (la2) at (.5,-1.1) {$i \leq y$ };
    \node (lamixed) at (3,-1.6) {$y - to\_int(y) < .3$};
  \end{scope}
  \begin{scope}[xshift=-2cm,yshift=5cm]
    \node (uf1) at (-.6,-.6) {$f(b) = v$};
    \node (uf2) at (.7,-1.1) {$f(a) \neq v$ };
  \end{scope}
  \begin{scope}[xshift=-5cm,yshift=4.5cm]
    \node (ar1) at (-.3,-.4) {$b = \store aiv$};
    \node (ar2) at (.1,-.9) {$\forall x. \select bx = i \rightarrow \select ax = v$};
  \end{scope}
  \begin{scope}[xshift=0cm,yshift=5cm]
    \node (lauf1) at (0, -1.5) {$\select b i \geq i$};
    \node (lauf2) at (-.9, -2.2) {$\forall x. f(x) + i = g(2v)$};
    \node (lauf3) at (2, -2) {$f(b) \leq i$};
  \end{scope}
  \coordinate (a1) at (-6,3.6);
  \coordinate (a2) at (6.5,3.6);
  \node[shape=cloud,cloud puffs=40,draw,inner sep=-1cm,aspect=6,
    callout relative pointer={(0,.4)},
    fit=(a1)(a2)
  ] {};
    \end{tikzpicture}
        
  \item that supports model/proof production and \emph{interpolation}, and\\
    \begin{tikzpicture}
      \node (x1) at (-7,0) {};
      \node[draw,shape=rectangle callout, callout relative pointer={(.5,.2)}] at (-5.5,0) {get-model};
      \node[draw,shape=rectangle callout, callout relative pointer={(-.4,.2)}] at (-2.4,0) {get-proof};
      \node[draw,shape=rectangle callout, callout relative pointer={(-.9,.2)}] at (0.4,0) {get-unsat-core};
      \node[draw,shape=rectangle callout, callout relative pointer={(-1.1,.2)}] at (4,0) {get-interpolants};
    \end{tikzpicture}
    
  \item (since 2019) has support for quantifiers
    (E-matching based unit/conflict propagation, MBQI).
  \end{itemize}

  \begin{center}
\begin{tikzpicture}
  \node[anchor=west] (url) at (-5,-.5)
    {\url{https://github.com/ultimate-pa/smtinterpol}};
  \node[anchor=west] (url) at (-5,-1)
    {\url{https://ultimate.informatik.uni-freiburg.de/smtinterpol}};
  \node[draw,shape=rectangle callout, callout relative pointer={(-1.2,-.2)}] at (6.7,0) {Open Source (LGPL)};
\end{tikzpicture}
  \end{center}
\end{frame}
\end{document}
