\documentclass[a4paper]{easychair}
\usepackage[english]{babel}
\usepackage{xspace}
\usepackage{hyperref}
\usepackage{bashful}

\newcommand\SI{SMTInterpol\xspace}
\newcommand{\version}{\splice{git describe}}
\newcommand{\TODO}[1]{\textcolor{red}{#1}}

\title{\SI\\{\large Version\version}}

\author{Jochen Hoenicke \and Tanja Schindler}
\institute{
  University of Freiburg\\
  \email{\{hoenicke,schindle\}@informatik.uni-freiburg.de}\\[1ex]
  \today
}
\titlerunning{\SI \version}
\authorrunning{Hoenicke and Schindler}

\begin{document}
%\enlargethispage{2em}
\maketitle
\section*{Description}
\SI is an SMT solver written in Java and available under LGPL v3.  It supports
the combination of the theories of uninterpreted functions,
linear arithmetic over integers and reals, and arrays.  Furthermore it can
produce models, proofs, unsatisfiable cores, and interpolants.  The solver
reads input in SMT-LIB format.  It includes parsers for DIMACS, AIGER, and
SMT-LIB version 1.2 and 2.5.

The solver is based on the well-known DPLL(T)/CDCL framework~\cite{DBLP:conf/cav/GanzingerHNOT04}.
It uses variants of standard algorithms for CNF
conversion~\cite{DBLP:journals/jsc/PlaistedG86} and congruence
closure~\cite{DBLP:conf/rta/NieuwenhuisO05}.
The solver for linear arithmetic is based on Simplex~\cite{DBLP:conf/cav/DutertreM06}, the sum-of-infeasibility algorithm~\cite{DBLP:conf/fmcad/KingBD13}, and branch-and-cut for integer arithmetic~\cite{DBLP:conf/cav/ChristH15,DBLP:conf/cav/DilligDA09}.
The array decision procedure is based on weak equivalences~\cite{DBLP:conf/frocos/ChristH15} and includes an extension for constant arrays~\cite{DBLP:conf/vmcai/HoenickeS19}.
Theory combination is performed based on partial models produced by the theory solvers~\cite{DBLP:journals/entcs/MouraB08}.

In the current release, the solver for quantified formulas was extensively revised.
The core of the new approach is an incremental search for instances of quantified clauses that result in a conflict or a propagation.
Instead of using E-matching as an instantiation heuristic, the E-graph is used to detect and evaluate candidate instances before adding them to the ground problem.
The solver decides the finite almost uninterpreted fragment~\cite{DBLP:conf/cav/GeM09}.

The main focus of \SI is the incremental track.
This track simulates the typical application of \SI where a user asks multiple queries.
As an extension \SI supports quantifier-free interpolation for the supported theories~\cite{DBLP:journals/jar/ChristH16,DBLP:conf/cade/HoenickeS18,DBLP:conf/vmcai/HoenickeS19}.
This makes it useful as a backend for software verification tools.  In particular,
\textsc{Ultimate Automizer}\footnote{\url{https://ultimate.informatik.uni-freiburg.de/}}
and \textsc{CPAchecker}\footnote{\url{https://cpachecker.sosy-lab.org/}}, the winners of the SV-COMP 2016--2020, use \SI.

\section*{Competition Version}
The version submitted to the SMT-COMP 2020 includes a revised solver for quantified formulas.
Proof production for quantified formulas is only partially implemented, and while this version supports unsat core extraction, it does not support interpolation in the presence of quantified formulas.
The solver is conservative and returns unknown for satisfiable formulas that are not in the supported fragment.

\section*{Webpage and Sources}
Further information about \SI can be found at
\begin{center}
  \url{http://ultimate.informatik.uni-freiburg.de/smtinterpol/}
\end{center}
The sources are available via GitHub
\begin{center}
  \url{https://github.com/ultimate-pa/smtinterpol}
\end{center}

\section*{Authors}
The code was written by J{\"u}rgen Christ, Matthias Heizmann, Jochen Hoenicke, Alexander Nutz, 
Markus Pomrehn, Pascal Raiola, and Tanja Schindler.

\section*{Logics, Tracks and Magic Number}

\SI participates in the single query track, the incremental track, the unsat core track, and the model validation track.
It supports all combinations of uninterpreted functions, linear arithmetic, and arrays, and participates in the following logic divisions:

ALIA, AUFLIA, AUFLIRA, LIA, LRA, QF\_ALIA, QF\_AUFLIA, QF\_AX,
QF\_IDL, QF\_LIA, QF\_LIRA, QF\_LRA, QF\_RDL, QF\_UF, QF\_UFIDL,
QF\_UFLIA, QF\_UFLRA, UF, UFIDL, UFLIA, UFLRA.

Magic Number: $192\,843\,011$

\bibliography{sysdec}
\bibliographystyle{alpha}
\end{document}
