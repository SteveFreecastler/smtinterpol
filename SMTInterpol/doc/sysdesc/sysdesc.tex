\documentclass[a4paper]{easychair}
\usepackage[english]{babel}
\usepackage{xspace}
\usepackage{hyperref}
\usepackage{bashful}

\newcommand\SI{SMTInterpol\xspace}
\newcommand{\version}{\splice{git describe}}
\newcommand{\TODO}[1]{\textcolor{red}{#1}}

\title{\SI\\{\large Version\version}}

\author{Jochen Hoenicke \and Tanja Schindler}
\institute{
  University of Freiburg\\
  \email{\{hoenicke,schindle\}@informatik.uni-freiburg.de}\\[1ex]
  \today
}
\titlerunning{\SI \version}
\authorrunning{Hoenicke and Schindler}

\begin{document}
%\enlargethispage{2em}
\maketitle
\section*{Description}
\SI is an SMT solver written in Java and available under LGPL v3.  It supports
the combination of the theories of uninterpreted functions,
linear arithmetic over integers and reals, and arrays.  Furthermore it can
produce models, proofs, unsatisfiable cores, and interpolants.  The solver
reads input in SMT-LIB format.  It includes parsers for DIMACS, AIGER, and
SMT-LIB version 1.2 and 2.5.

The solver is based on the well-known DPLL(T)/CDCL framework~\cite{DBLP:conf/cav/GanzingerHNOT04}.
The solver uses variants of standard algorithms for CNF
conversion~\cite{DBLP:journals/jsc/PlaistedG86}, congruence
closure~\cite{DBLP:conf/rta/NieuwenhuisO05}, Simplex~\cite{DBLP:conf/cav/DutertreM06} and
branch-and-cut for integer arithmetic~\cite{DBLP:conf/cav/ChristH15,DBLP:conf/cav/DilligDA09}.
The array decision procedure is based on weak equivalences~\cite{DBLP:conf/frocos/ChristH15}.
Theory combination is performed based on partial models produced by the theory
solvers~\cite{DBLP:journals/entcs/MouraB08}.

This release uses the ``sum-of-infeasibility'' algorithm~\cite{DBLP:conf/fmcad/KingBD13} for linear arithmetic.
The array theory was extended by constant arrays~\cite{DBLP:conf/vmcai/HoenickeS19}.
New in the current release is an experimental solver for quantified formulas.
It is based on model-based quantifier instantiation, in particular, it solves the almost uninterpreted fragment~\cite{DBLP:conf/cav/GeM09}.
This approach is combined with ideas from conflict-based quantifier instantiation~\cite{DBLP:conf/fmcad/ReynoldsTM14}.
A distinguishing feature is that it does not use pattern-based E-matching at all, but uses the E-graph to identify potential conflict and unit clauses.

The main focus of \SI is the incremental track.
This track simulates the typical
application of \SI where a user asks multiple queries.  The main focus
of the development team of \SI is the interpolation
engine~\cite{DBLP:journals/jar/ChristH16,DBLP:conf/cade/HoenickeS18}.
This makes it useful as a backend for software verification tools.  In particular,
\textsc{Ultimate Automizer}\footnote{\url{https://ultimate.informatik.uni-freiburg.de/}} %~\cite{DBLP:conf/tacas/HeizmannCDGHLNM18}
and \textsc{CPAchecker}\footnote{\url{https://cpachecker.sosy-lab.org/}}%~\cite{DBLP:conf/tacas/DanglLW15}
, the winners of the SV-COMP 2016--2019, use \SI.

\section*{Competition Version}
The version submitted to the SMT-COMP 2019 is an experimental release with the new solver for quantified formulas.
The algorithms to generate proofs and models are not adapted to the new quantifier theory.
Therefore, some features like unsat cores do not work yet.
The solver is conservative and returns unknown for satisfiable formulas that are not in the supported fragment.

\section*{Webpage and Sources}
Further information about \SI can be found at
\begin{center}
  \url{http://ultimate.informatik.uni-freiburg.de/smtinterpol/}
\end{center}
The sources are available via GitHub
\begin{center}
  \url{https://github.com/ultimate-pa/smtinterpol}
\end{center}

\section*{Authors}
The code was written by J{\"u}rgen Christ, Matthias Heizmann, Jochen Hoenicke, Alexander Nutz, 
Markus Pomrehn, Pascal Raiola, and Tanja Schindler.

\section*{Logics, Tracks and Magic Number}

\SI participates in the single-query track, the incremental track, and the
unsat core track.  In the single-query and incremental track it supports all
combinations of uninterpreted functions, linear arithmetic, and
arrays: ALIA, AUFLIA, AUFLIRA, LIA, LRA, QF\_ALIA, QF\_AUFLIA, QF\_AX,
QF\_IDL, QF\_LIA, QF\_LIRA, QF\_LRA, QF\_RDL, QF\_UF, QF\_UFIDL,
QF\_UFLIA, QF\_UFLRA, UF, UFIDL, UFLIA, UFLRA.
In the unsat core track, \SI participates only in the quantifier-free logics:
QF\_ALIA, QF\_AUFLIA, QF\_AX,
QF\_IDL, QF\_LIA, QF\_LIRA, QF\_LRA, QF\_RDL, QF\_UF, QF\_UFIDL,
QF\_UFLIA, QF\_UFLRA.

Magic Number: $983\,571\,724$

\bibliography{sysdec}
\bibliographystyle{alpha}
\end{document}
